\section{Forces and Moments}
To model how the quadcopter interacts with its environment, we have derived the forces and moments that act upon it from first principles. The split of the forces into internal and external is a purely conceptual one.
\subsection{External forces}
The quadcopter chiefly experiences two external forces: the force of gravitational attraction between itself and the Earth and the retarding force of air friction or drag.
\subsubsection{Gravity $\vec{F}_g$}
The simplifications used in the derivation of the gravitational force are as follows: gravity was taken to have a constant magnitude of  \SI{9.81}{\metre\per\second\squared} irrespective of altitude.
\\
The gravitational force in the body-fixed frame was obtained as follows:
$$
\colvec{F}_g^b = \mat{Q}\cdot\colvec{F}_g^i = \begin{bmatrix}
m\,g\,\sin(\theta)\\-\,m\,g\,\sin(\phi)\cos(\theta)\\-\,m\,g\,\cos(\phi)\cos(\theta)
\end{bmatrix}
\mathrm{, where }\:
\colvec{F}_g^i = \begin{bmatrix}
0\\0\\9.81
\end{bmatrix}
\si{\metre\per\second\squared}
$$
\subsubsection{Drag $\vec{F}_d$}
The force of air friction or drag was modelled as an opposing force to the given direction of motion. The simplifications involved were: taking the density of air to be a constant value, irrespective of temperature and altitude; using a model that doesn't take into account the actual airflow over the drone; disregarding any turbulence caused by the propellers. The formula is as follows~:
\begin{equation}
\label{eqn:drag}
\vec{F}_d = \frac{1}{2}\:C_d\:\rho_{air}\:u^2\:\Delta_f
\end{equation}
, where $C_d$ is a coefficient of drag\footnote{This was determined by taking the top speed of the drone from its spec \cite{drone-spec} and|having estimated a frontal area as outlined above|working back from there to establish an estimate. }, $\rho_{air}$ is the (constant) density of air, $u$ is the magnitude of the quadcopter's velocity in the direction of motion, $\Delta_f$ is the frontal area of the quadcopter with respect to the direction of motion.

The frontal area for \cref{eqn:drag} was established by modelling the quadcopter as a box translating and rotating through space.
% TODO put in more about the box model, mention bff
\subsection{Internal forces}
As the quadcopter was modelled as a rigid body (see \cref{rigid-body-motion}) we acknowledge no internal body forces such as shear or torsion. This section deals with the forces that are most convenient to treat in the body-fixed frame, i.e. the thrust developed by the propellers.
\subsubsection[Thrust]{Thrust  }
\subsection{Moments}