\section{Theoretical Background}



\subsection{Frames of reference}

\subsection{Notation} %Should this be an appendix/epilogue/monologue/preview????
-> Define mapping from vectors to column matrices

\subsection{Analysis of the System Configuration}\label{s:analysis}
\begin{figure*}[tb]
	\centering
	\begin{tikzpicture}[scale=1]
\coordinate (origin) at (0,0);
\coordinate (com) at (\thetasep:\baxissep);
%DRONE
\node [draw, color=gray, shape=rectangle, minimum width=\dronew cm, minimum height=\droneh cm, anchor=center, rotate around={-\thetayy:(com)}] at (com) {};

% Ei
\draw [->,thick] (origin) -- (0,5) node (ei3) [above] {\axis{i}{3}};
\draw [->,thick] (origin) -- ++(5,0) node (ei1) [right] {\axis{i}{1}};
% Eb
\draw [->,thick] (com) -- +(90-\thetayy:\baxislen) node (eb3) [above] {\axis{b}{3}};
\draw [->,thick] (com) -- +(-\thetayy:\baxislen) node (eb1) [right] {\axis{b}{1}};

% PITCH ANGLE
\draw [dashed] (com) -- (com -| origin);
\draw [-stealth,dashed] (com) ++(-\dronew/3,0) coordinate(parcs)  arc(180:180-\thetayy:\dronew/3)  coordinate (parce);
\path (com) -- (parcs) node[pos=0.5,anchor=south east] {$\scriptstyle\theta$};
\draw [dashed] (com) -- (parce);
\end{tikzpicture}
	\caption{System configuration}
	\label{fig:sysconfig}
\end{figure*}
\begin{figure*}[tb]
	\centering
	%set the plot display orientation
%syntax: \tdplotsetdisplay{\theta_d}{\phi_d}
\tdplotsetmaincoords{60}{180-45}

%define polar coordinates for some vector
%TODO: look into using 3d spherical coordinate system
\pgfmathsetmacro{\rvec}{\baxissep}
\pgfmathsetmacro{\thetavec}{45}
\pgfmathsetmacro{\phivec}{52}
\def\phib{6}
\def\thetab{60}
\def\psib{25}
\def\thrh{0.5}
\usetikzlibrary{3d}
\makeatletter
\tikzoption{canvas is plane}[]{\@setOxy#1}
\def\@setOxy O(#1,#2,#3)x(#4,#5,#6)y(#7,#8,#9)%
{\def\tikz@plane@origin{\pgfpointxyz{#1}{#2}{#3}}%
	\def\tikz@plane@x{\pgfpointxyz{#4}{#5}{#6}}%
	\def\tikz@plane@y{\pgfpointxyz{#7}{#8}{#9}}%
	\tikz@canvas@is@plane
}
\makeatother  
%start tikz picture, and use the tdplot_main_coords style to implement the display 
%coordinate transformation provided by 3dplot
\begin{tikzpicture}[scale=1,tdplot_main_coords]
%set up some coordinates 
%-----------------------
\coordinate (O) at (0,0,0);

%determine a coordinate (P) using (r,\theta,\phi) coordinates.  This command
%also determines (Pxy), (Pxz), and (Pyz): the xy-, xz-, and yz-projections
%of the point (P).
%syntax: \tdplotsetcoord{Coordinate name without parentheses}{r}{\theta}{\phi}
\tdplotsetcoord{P}{\rvec}{\thetavec}{\phivec}

%draw figure contents
%--------------------

%draw the main coordinate system axes
\draw[thick,->] (0,0,0) -- (5,0,0) node[anchor=north]{\axis{i}{1}};
\draw[thick,->] (0,0,0) -- (0,5,0) node[anchor=west]{\axis{i}{2}};
\draw[thick,->] (0,0,0) -- (0,0,5) node[anchor=south]{\axis{i}{3}};

%draw a vector from origin to point (P) 
%\draw[-stealth,color=red] (O) -- (P);

%draw projection on xy plane, and a connecting line
%\draw[dashed, color=red] (O) -- (Pxy);
%\draw[dashed, color=red] (P) -- (Pxy);

%draw the angle \phi, and label it
%syntax: \tdplotdrawarc[coordinate frame, draw options]{center point}{r}{angle}{label options}{label}
%\tdplotdrawarc{(O)}{0.2}{0}{\phivec}{anchor=north}{$\phi$}


%set the rotated coordinate system so the x'-y' plane lies within the
%"theta plane" of the main coordinate system
%syntax: \tdplotsetthetaplanecoords{\phi}
\tdplotsetthetaplanecoords{\phivec}

%draw theta arc and label, using rotated coordinate system
%\tdplotdrawarc[tdplot_rotated_coords]{(0,0,0)}{0.5}{0}{\thetavec}{anchor=south west}{$\theta$}

%draw some dashed arcs, demonstrating direct arc drawing
%\draw[dashed,tdplot_rotated_coords] (\rvec,0,0) arc (0:90:\rvec);
%\draw[dashed] (\rvec,0,0) arc (0:90:\rvec);

%set the rotated coordinate definition within display using a translation
%coordinate and Euler angles in the "z(\alpha)y(\beta)z(\gamma)" euler rotation convention
%syntax: \tdplotsetrotatedcoords{\alpha}{\beta}{\gamma}
\tdplotsetrotatedcoords{\thetab}{\phib}{\psib}
%\tdplotsetrotatedcoords{\phivec}{\thetave}{0}

%translate the rotated coordinate system
%syntax: \tdplotsetrotatedcoordsorigin{point}
\tdplotsetrotatedcoordsorigin{(P)}

\begin{scope}[tdplot_rotated_coords,canvas is plane={O(0,0,0)x(1,0,0)y(0,1,0)}]
\draw [fill=red] (-\droned/2,-\dronew/2) rectangle (\droned/2,\dronew/2);
\draw [fill=green] (-\dronew/2,-\droned/2) rectangle (\dronew/2,\droned/2) ;
\end{scope}
\begin{scope}[tdplot_rotated_coords,canvas is plane={O(0,0,0)x(0,1,0)y(0,0,1)},transform shape]
\draw [-stealth,ultra thick,blue]  (-\dronew/2+\droned/2,0) -- ++(0,\thrh) node [above] {\textbf{$T_2$}};
\draw [-stealth,ultra thick,blue]  (\dronew/2-\droned/2,0) -- ++(0,\thrh) node [above] {\textbf{$T_4$}};
\end{scope}
\begin{scope}[tdplot_rotated_coords,canvas is plane={O(0,0,0)x(1,0,0)y(0,0,1)},transform shape]
\draw [-stealth,ultra thick,blue]  (-\dronew/2+\droned/2,0) -- ++(0,\thrh) node [above] {\textbf{$T_3$}};
\draw [-stealth,ultra thick,blue]  (\dronew/2-\droned/2,0) -- ++(0,\thrh) node [above] {\textbf{$T_1$}};
\end{scope}
%use the tdplot_rotated_coords style to work in the rotated, translated coordinate frame
\draw[thick,tdplot_rotated_coords,->] (0,0,0) -- (\baxislen,0,0) node[anchor=north west]{\axis{b}{1}};
\draw[thick,tdplot_rotated_coords,->] (0,0,0) -- (0,\baxislen,0) node[anchor=west]{\axis{b}{2}};
\draw[thick,tdplot_rotated_coords,->] (0,0,0) -- (0,0,\baxislen) node[anchor=south]{\axis{b}{3}};

%WARNING:  coordinates defined by the \coordinate command (eg. (O), (P), etc.)
%cannot be used in rotated coordinate frames.  Use only literal coordinates.  

%draw some vector, and its projection, in the rotated coordinate frame
%\draw[-stealth,color=blue,tdplot_rotated_coords] (0,0,0) -- (.2,.2,.2);

%\draw[dashed,color=blue,tdplot_rotated_coords] (0,0,0) -- (.2,.2,0);
%\draw[dashed,color=blue,tdplot_rotated_coords] (.2,.2,0) -- (.2,.2,.2);

%show its phi arc and label
%\tdplotdrawarc[tdplot_rotated_coords,color=blue]{(0,0,0)}{0.2}{0}{45}{anchor=north west,color=black}{$\phi'$}

%change the rotated coordinate frame so that it lies in its theta plane.
%Note that this overwrites the original rotated coordinate frame
%syntax: \tdplotsetrotatedthetaplanecoords{\phi'}
\tdplotsetrotatedthetaplanecoords{45}

%draw theta arc and label
%\tdplotdrawarc[tdplot_rotated_coords,color=blue]{(0,0,0)}{0.2}{0}{55}{anchor=south west,color=black}{$\theta'$}

\end{tikzpicture}
	\caption{System in Choi Ahn configuration}
	\label{fig:choiahnconfig}
\end{figure*}
In order to simplify the assignment we first analysed the configuration detailed in the task.

\Cref{fig:choiahnconfig} depicts a system in the Choi/Ahn configuration. The system has 6 degrees of freedom. Our task however, requires a configuration with only 3 degrees of freedom, namely movement in the \axis{b}{1} and \axis{b}{3} axes and pitch|angular rotation about the \axis{b}{2} axis.
Assuming that under initial conditions, the body of the drone and its axes are aligned with the axes of the inertial frame, the requirements for a 3-dimensional representation of translational movement can be reduced to a 2-dimensional one. Even if the axes are aligned under starting conditions a rotation about the \axis{b}{1} or \axis{b}{3} axes (roll and yaw respectively or $\phi$ and $\psi$) would rotate the vehicle in a 3-dimensional space; in this instance, this will not be considered as the arising of torques about the \axis{b}{1} or \axis{b}{3} axes is not possible given the problem description.

\Cref{fig:sysconfig} depicts the configuration of the reduced system for the given task. The vehicle can move anywhere inside this two dimensional space.

\subsection{Mapping from Inertial to Body-Fixed Reference Frame}
%% TIARNACH: Moved this to an appendix
%\subsection{Model Assumptions}

\subsection{State-Space Realisation}
The objective of this task is to derive a linearised dynamic system describing the motion of a quadcopter in the Choi/Ahn configuration. Our approach to the task began with developing a non-linear state-space system. There are 12 state quantities in the Choi/Ahn configuration; namely x, y, z coordinates to describe the drone's position in the inertial frame, linear and angular velocities in the body-fixed frame, and Euler angles. The state vector (\cref{eqn:state def}) is defined by these 12 quantities and the state derivative (\cref{eqn:ddt state def}) by the time derivatives of these 12 quantities.
\begin{captioneq}[htb]
	\begin{equation}\label{eqn:state def}
\mstate=\begin{bmatrix}u\\v\\w\\p\\q\\r\\\cmidrule(lr){1-1 } x\\y\\z\\\phi\\\theta\\\psi\end{bmatrix}
\end{equation}
\caption{State vector}
\end{captioneq}
\begin{captioneq}[htb]
\begin{equation}\label{eqn:ddt state def}
\mdstate=\begin{bmatrix}\dot{u}\\\dot{v}\\\dot{w}\\\dot{p}\\\dot{q}\\\dot{r}\\\cmidrule(lr){1-1} \dot{x}\\\dot{y}\\\dot{z}\\\dot{\phi}\\\dot{\theta}\\\dot{\psi}\end{bmatrix}
\end{equation}
\caption{State derivative}
\end{captioneq}


LITTLE DESCRIPTION FOR THE MATRIX - TOP HALF IS BODY FIXED, BOTTOM HALF IS INERTIAL

CIAN TO CIAN: PUT IN A FULL PICTURE OF THE STATE MATRIX

Applying the 2-dimensional model derived in \cref{s:analysis} we can reduce the state matrix to the applicable quantities. \Cref{eqn:state reduced} is the reduced state vector and its time derivative.

REDUCED STATE VECTOR AND STATE DERIVATIVE

Our next objective was to derive a non-linear system of equations for the configuration.

\subsubsection[Inertial frame]{Inertial frame \Frame{i}} % Necessary? Nope
\subsubsection[Body-fixed frame]{Body-fixed frame \Frame{b}}
Axes \axis{b}{1}, \axis{b}{2} and \axis{b}{3}.
\subsubsection[Linear Mapping Matrix]{Linear Mapping Matrix \mat{Q}}
\label{sss:lmapmatQ}

\subsection{Quadcopter Dynamics}
\subsubsection{Rigid body motion}
\label{rigid-body-motion}