\section{Theoretical Background}



\subsection{Frames of reference}

\subsection{Notation} %Should this be an appendix/epilogue/monologue/preview????
-> Define mapping from vectors to column matrices

\subsection{Analysis of the System Configuration}\label{s:analysis}
\begin{figure*}[tb]
	\centering
	\def\thetayy{30}
\def\iaxislen{5}
\def\baxislen{2}
\def\baxissep{4}
\def\thetasep{45}
\def\dronew{3}
\def\droneh{0.5}
\begin{tikzpicture}[scale=1]
\coordinate (origin) at (0,0);
\coordinate (com) at (\thetasep:\baxissep);
%DRONE
\node [draw, fill, color=gray, thin, shape=rectangle, minimum width=\dronew cm, minimum height=\droneh cm, anchor=center, rotate around={-\thetayy:(com)}] at (com) {};

% Ei
\draw [->,thick] (origin) -- (0,5) node (ei3) [above] {\axis{i}{3}};
\draw [->,thick] (origin) -- ++(5,0) node (ei1) [right] {\axis{i}{1}};
% Eb
\draw [->] (com) -- +(90-\thetayy:\baxislen) node (eb3) [above] {\axis{b}{3}};
\draw [->] (com) -- +(-\thetayy:\baxislen) node (eb1) [right] {\axis{b}{1}};

% PITCH ANGLE
\draw [dashed] (com) -- (com -| origin);
\draw [->,dashed] (com) ++(-\dronew/3,0) coordinate(parcs)  arc(180:180-\thetayy:\dronew/3)  coordinate (parce);
\path (com) -- (parcs) node[pos=0.5,anchor=south east] {$\scriptstyle\theta$};
\draw [dashed] (com) -- (parce);
\end{tikzpicture}
	\caption{System configuration}
	\label{fig:sysconfig}
\end{figure*}
\begin{figure*}[tb]
	\centering
	\input{img/choi_ahn_config}
	\caption{System in Choi Ahn configuration}
	\label{fig:choiahnconfig}
\end{figure*}
In order to simplify the assignment we first analysed the configuration detailed in the task.

\Cref{fig:choiahnconfig} depicts a system in the Choi/Ahn configuration. The system has 6 degrees of freedom. Our task however, requires a configuration with only 3 degrees of freedom, namely movement in the \axis{b}{1} and \axis{b}{3} axes and pitch|angular rotation about the \axis{b}{2} axis.
Assuming that under initial conditions, the body of the drone and its axes are aligned with the axes of the inertial frame, the requirements for a 3-dimensional representation of translational movement can be reduced to a 2-dimensional one. Even if the axes are aligned under starting conditions a rotation about the \axis{b}{1} or \axis{b}{3} axes (roll and yaw respectively or $\phi$ and $\psi$) would rotate the vehicle in a 3-dimensional space; in this instance, this will not be considered as the arising of torques about the \axis{b}{1} or \axis{b}{3} axes is not possible given the problem description.

\Cref{fig:sysconfig} depicts the configuration of the reduced system for the given task. The vehicle can move anywhere inside this two dimensional space.

\subsection{Mapping from Inertial to Body-Fixed Reference Frame}
%% TIARNACH: Moved this to an appendix
%\subsection{Model Assumptions}

\subsection{State-Space Realisation}
The objective of this task is to derive a linearised dynamic system describing the motion of a quadcopter in the Choi/Ahn configuration. Our approach to the task began with developing a non-linear state-space system. There are 12 state quantities in the Choi/Ahn configuration; namely x, y, z coordinates to describe the drone's position in the inertial frame, linear and angular velocities in the body-fixed frame, and Euler angles. The state vector (\cref{eqn:state def}) is defined by these 12 quantities and the state derivative (\cref{eqn:ddt state def}) by the time derivatives of these 12 quantities.
\begin{captioneq}[htb]
	\begin{equation}\label{eqn:state def}
\mstate=\begin{bmatrix}u\\v\\w\\p\\q\\r\\\cmidrule(lr){1-1 } x\\y\\z\\\phi\\\theta\\\psi\end{bmatrix}
\end{equation}
\caption{State vector}
\end{captioneq}
\begin{captioneq}[htb]
\begin{equation}\label{eqn:ddt state def}
\mdstate=\begin{bmatrix}\dot{u}\\\dot{v}\\\dot{w}\\\dot{p}\\\dot{q}\\\dot{r}\\\cmidrule(lr){1-1} \dot{x}\\\dot{y}\\\dot{z}\\\dot{\phi}\\\dot{\theta}\\\dot{\psi}\end{bmatrix}
\end{equation}
\caption{State derivative}
\end{captioneq}


LITTLE DESCRIPTION FOR THE MATRIX - TOP HALF IS BODY FIXED, BOTTOM HALF IS INERTIAL

CIAN TO CIAN: PUT IN A FULL PICTURE OF THE STATE MATRIX

Applying the 2-dimensional model derived in \cref{s:analysis} we can reduce the state matrix to the applicable quantities. \Cref{eqn:state reduced} is the reduced state vector and its time derivative.

REDUCED STATE VECTOR AND STATE DERIVATIVE

Our next objective was to derive a non-linear system of equations for the configuration.

\subsubsection[Inertial frame]{Inertial frame \Frame{i}} % Necessary? Nope
\subsubsection[Body-fixed frame]{Body-fixed frame \Frame{b}}
Axes \axis{b}{1}, \axis{b}{2} and \axis{b}{3}.
\subsubsection[Linear Mapping Matrix]{Linear Mapping Matrix \mat{Q}}
\label{sss:lmapmatQ}

\subsection{Quadcopter Dynamics}
\subsubsection{Rigid body motion}
\label{rigid-body-motion}