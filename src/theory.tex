\section{Theoretical Background}

\subsection{Frames of reference}

\subsection{Notation} %Should this be an apendix/epilogue/monologue/preview????
-> Define mapping from vectors to column matrices
\subsection{Configuration}
Diagram of the configuration

\subsection{Space State Realisation}
The objective of this task is to derive a linearised dynamic system describing the motion of a quadcopter in the Choi/Ahn configuration. It is therefore logical to develop a space-state representation of the system. There are 12 state quantities of interest in the given configuration; namely the linear and angular velocities of the inertial reference frame and the body-fixed frame respectively. The state vector is defined by these 12 quantities and their respective time derivatives. Given (((below/to the left/to the right???))) is the state vector.

Little description for the matrix: Quantities in the top half are bf-frame, bottom are inertial - separated by line

Cian to Cian: put in a picture of the full state matrix
\subsubsection{Simplifications}
The scope of the task does not require the drone's position and velocity in inertial space, but rather the the body-fixed velocities, angular velocities, and Euler angles. Furthermore, analysis of the system configuration reveals that linear motion is limited to the E1 and E3 direction, and angular only about the E2 axis. See fig 1.wallawalla .

Accounting for these simplifications results in a simplified state vector. For the scope of this task, we have only derived equations of motion for the following state variables.
\subsubsection[Inertial frame]{Inertial frame \Frame{i}} % Necessary?
\subsubsection[Body-fixed frame]{Body-fixed frame \Frame{b}}
Axes \axis{b}{1}, \axis{b}{2} and \axis{b}{3}.
\subsubsection[Linear Mapping Matrix]{Linear Mapping Matrix \mat{Q}}
\label{sss:lmapmatQ}

\subsection{Quadcopter Dynamics}
\subsubsection{Rigid body motion}
\label{rigid-body-motion}