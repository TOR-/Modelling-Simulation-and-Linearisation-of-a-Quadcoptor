%set the plot display orientation
%syntax: \tdplotsetdisplay{\theta_d}{\phi_d}
\tdplotsetmaincoords{60}{180-45}

%define polar coordinates for some vector
%TODO: look into using 3d spherical coordinate system
\pgfmathsetmacro{\rvec}{\baxissep}
\pgfmathsetmacro{\thetavec}{45}
\pgfmathsetmacro{\phivec}{52}
\def\phib{6}
\def\thetab{60}
\def\psib{25}
\def\thrh{0.5}
\usetikzlibrary{3d}
\makeatletter
\tikzoption{canvas is plane}[]{\@setOxy#1}
\def\@setOxy O(#1,#2,#3)x(#4,#5,#6)y(#7,#8,#9)%
{\def\tikz@plane@origin{\pgfpointxyz{#1}{#2}{#3}}%
	\def\tikz@plane@x{\pgfpointxyz{#4}{#5}{#6}}%
	\def\tikz@plane@y{\pgfpointxyz{#7}{#8}{#9}}%
	\tikz@canvas@is@plane
}
\makeatother  
%start tikz picture, and use the tdplot_main_coords style to implement the display 
%coordinate transformation provided by 3dplot
\begin{tikzpicture}[scale=1,tdplot_main_coords]

%set up some coordinates 
%-----------------------
\coordinate (O) at (0,0,0);

%determine a coordinate (P) using (r,\theta,\phi) coordinates.  This command
%also determines (Pxy), (Pxz), and (Pyz): the xy-, xz-, and yz-projections
%of the point (P).
%syntax: \tdplotsetcoord{Coordinate name without parentheses}{r}{\theta}{\phi}
\tdplotsetcoord{P}{\rvec}{\thetavec}{\phivec}

%draw figure contents
%--------------------

%draw the main coordinate system axes
\draw[thick,->] (0,0,0) -- (5,0,0) node[anchor=north]{\axis{i}{1}};
\draw[thick,->] (0,0,0) -- (0,5,0) node[anchor=west]{\axis{i}{2}};
\draw[thick,->] (0,0,0) -- (0,0,5) node[anchor=south]{\axis{i}{3}};

%draw a vector from origin to point (P) 
%\draw[-stealth,color=red] (O) -- (P);

%draw projection on xy plane, and a connecting line
%\draw[dashed, color=red] (O) -- (Pxy);
%\draw[dashed, color=red] (P) -- (Pxy);

%draw the angle \phi, and label it
%syntax: \tdplotdrawarc[coordinate frame, draw options]{center point}{r}{angle}{label options}{label}
%\tdplotdrawarc{(O)}{0.2}{0}{\phivec}{anchor=north}{$\phi$}


%set the rotated coordinate system so the x'-y' plane lies within the
%"theta plane" of the main coordinate system
%syntax: \tdplotsetthetaplanecoords{\phi}
\tdplotsetthetaplanecoords{\phivec}

%draw theta arc and label, using rotated coordinate system
%\tdplotdrawarc[tdplot_rotated_coords]{(0,0,0)}{0.5}{0}{\thetavec}{anchor=south west}{$\theta$}

%draw some dashed arcs, demonstrating direct arc drawing
%\draw[dashed,tdplot_rotated_coords] (\rvec,0,0) arc (0:90:\rvec);
%\draw[dashed] (\rvec,0,0) arc (0:90:\rvec);

%set the rotated coordinate definition within display using a translation
%coordinate and Euler angles in the "z(\alpha)y(\beta)z(\gamma)" euler rotation convention
%syntax: \tdplotsetrotatedcoords{\alpha}{\beta}{\gamma}
\tdplotsetrotatedcoords{\thetab}{\phib}{\psib}
%\tdplotsetrotatedcoords{\phivec}{\thetave}{0}

%translate the rotated coordinate system
%syntax: \tdplotsetrotatedcoordsorigin{point}
\tdplotsetrotatedcoordsorigin{(P)}

\begin{scope}[tdplot_rotated_coords,canvas is plane={O(0,0,0)x(1,0,0)y(0,1,0)}]
\draw [fill=red] (-\droned/2,-\dronew/2) rectangle (\droned/2,\dronew/2);
\draw [fill=green] (-\dronew/2,-\droned/2) rectangle (\dronew/2,\droned/2) ;
\end{scope}
\begin{scope}[tdplot_rotated_coords,canvas is plane={O(0,0,0)x(0,1,0)y(0,0,1)},transform shape]
\draw [-stealth,ultra thick,blue]  (-\dronew/2+\droned/2,0) -- ++(0,\thrh) node [above] {\textbf{$T_2$}};
\draw [-stealth,ultra thick,blue]  (\dronew/2-\droned/2,0) -- ++(0,\thrh) node [above] {\textbf{$T_4$}};
\end{scope}
\begin{scope}[tdplot_rotated_coords,canvas is plane={O(0,0,0)x(1,0,0)y(0,0,1)},transform shape]
\draw [-stealth,ultra thick,blue]  (-\dronew/2+\droned/2,0) -- ++(0,\thrh) node [above] {\textbf{$T_3$}};
\draw [-stealth,ultra thick,blue]  (\dronew/2-\droned/2,0) -- ++(0,\thrh) node [above] {\textbf{$T_1$}};
\end{scope}
%use the tdplot_rotated_coords style to work in the rotated, translated coordinate frame
\draw[thick,tdplot_rotated_coords,->] (0,0,0) -- (\baxislen,0,0) node[anchor=north west]{\axis{b}{1}};
\draw[thick,tdplot_rotated_coords,->] (0,0,0) -- (0,\baxislen,0) node[anchor=west]{\axis{b}{2}};
\draw[thick,tdplot_rotated_coords,->] (0,0,0) -- (0,0,\baxislen) node[anchor=south]{\axis{b}{3}};

%WARNING:  coordinates defined by the \coordinate command (eg. (O), (P), etc.)
%cannot be used in rotated coordinate frames.  Use only literal coordinates.  

%draw some vector, and its projection, in the rotated coordinate frame
%\draw[-stealth,color=blue,tdplot_rotated_coords] (0,0,0) -- (.2,.2,.2);

%\draw[dashed,color=blue,tdplot_rotated_coords] (0,0,0) -- (.2,.2,0);
%\draw[dashed,color=blue,tdplot_rotated_coords] (.2,.2,0) -- (.2,.2,.2);

%show its phi arc and label
%\tdplotdrawarc[tdplot_rotated_coords,color=blue]{(0,0,0)}{0.2}{0}{45}{anchor=north west,color=black}{$\phi'$}

%change the rotated coordinate frame so that it lies in its theta plane.
%Note that this overwrites the original rotated coordinate frame
%syntax: \tdplotsetrotatedthetaplanecoords{\phi'}
\tdplotsetrotatedthetaplanecoords{45}

%draw theta arc and label
%\tdplotdrawarc[tdplot_rotated_coords,color=blue]{(0,0,0)}{0.2}{0}{55}{anchor=south west,color=black}{$\theta'$}

\end{tikzpicture}